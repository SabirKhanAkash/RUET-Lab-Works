\documentclass[preprint,11pt]{elsarticle}
\usepackage{amsmath}
\usepackage{graphicx}
\usepackage{booktabs,subcaption,amsfonts,dcolumn}
\usepackage{caption}
\usepackage{amssymb}
\usepackage{times,amsmath,epsfig}
\journal{Submitted for the Journal}

\begin{document}
	
	\begin{frontmatter}
		\title{General Drafts}
		\author{Md Shabir Khan Akash\\Roll : 1603108\\Department of Computer Science and Engineering\\Email : 4175.akash@gmail.com
		}
		\begin{abstract}
			A good abstract should answer the following few
			questions. Question-1 what is the general topic of the article,
			Question-2 what is the specific topic, Question-3 what is the
			research problem, Question-4 what is the current status of the
			problem and finally Question-5 what is the contribution(s) of the
			article and/or work.
		\end{abstract}
		
		\begin{keyword}
			Title, Introduction, Methodology,conclusion, reference
		\end{keyword}
		
	\end{frontmatter}
	
	
	\section{Introduction}
	\label{sec:introduction}
	This section is used to give little background and motivation
	why to write this paper/article. More specifically a detail
	answer of the questions from the abstract section
	\section{METHODOLOGY}
	To provide a framework/taxonomy to define the scope and
	methodology, technical notations that you are going to use.
	
	\section{MAIN BODY}
	To list, describe and compare the leading work in the areas
	using the uniform survey method/style that your have defined
	in the above section. This section may contain some definition
	like as follows:\\
	\textsc{Definition 1 (Composition anonymity):}\textit{For an individual i,
		the composition anonymity offered by n independent k-
		anonymized data sets is equal to the number of distinct
		common sensitive values of the equivalence classes in which
		the individual’s record resides.}\\
	\noindent\rule{4cm}{0.4pt}\\
	$\ast$Dr. Trovato insisted his name be rst.\\
	$\dagger$The secretary disavows any knowledge of this author's ac-
	tions.\\
	$\ddagger$This author is the one who did all the really hard work.
	\subsection{Part-1}
	The numbers of sections and subsections are subject to your topic. This section may contain some equations like follows:
	
	\begin{align}
	P(\hat{t}) &= P(\hat{q_1})\times P(\hat{q_{2}})\times ...\times P(\hat{q_{m}})\times P(s) \nonumber \\
	&= (\prod_{i=1}^{m} P(\hat{q_{i}}))\times P(s)
	\end{align}
	
	\subsection{Part-2}
	In end of each section and end of this paper, it is always a
	good ideas to summarize your work by listing the technologies/
	methods that you have discussed and compare them using
	a table or figure.
	\subsection{Part-2}
	In end of each section and end of this paper, it is always a
	good ideas to summarize your work by listing the technologies/
	methods that you have discussed and compare them using
	a table or figure.
	
	\begin{table}[h]
		\begin{center}
			\caption{COMMON NOTATION USED HERE}
			\label{notation}
			\begin{tabular}{ |l|l| } 
				\hline
				\textbf{Notation} & \textbf{Description}  \\ 
				\hline
				$\Omega$ & large population \\
				\hline 
				$D1, D_{1}, D_{2}$ & the original data sets\\
				\hline
				$ S^{d} $ & the set of \textit{d} different sensitive values\\
				\hline
				
			\end{tabular}
		\end{center}
	\end{table}
	\vfill
	
	\newpage
	\subsubsection{Table}
	Draw a table as follows:
	\begin{table}[h]
		\begin{subtable}{0.55\textwidth}
			\centering
			\begin{tabular}[h!]{|c|c|c|c|}
				\hline
				x & Method 1 & Method 2 & Method 3 \\
				\hline
				5 & 100 & 103 & 98  \\
				\hline
				10 & 102 & 50 & 80  \\
				\hline
				15 & 103 & 70 & 10  \\
				\hline
				20 & 20 & 50 & 50  \\
				\hline
			\end{tabular}
			\captionsetup{justification=centering}
			\caption{Table 1}
			\label{tab:table1_a}
		\end{subtable}%
		\hspace{\fill}
		\begin{subtable}{0.55\textwidth}
			\centering
			\begin{tabular}[h!]{|c|c|c|c|}
				\hline
				x & Method 1 & Method 2 & Method 3 \\
				\hline
				5 & 100 & 103 & 98  \\
				\hline
				10 & 102 & 50 & 80  \\
				\hline
				15 & 103 & 70 & 10  \\
				\hline
				20 & 20 & 50 & 50  \\
				\hline
			\end{tabular}
			
			\caption{Table 2}
			\label{tab:table1_a}
		\end{subtable}%
		\caption{}
	\end{table}
	\subsubsection{Graph}
	\begin{figure}[h!]
		\centering
		\begin{subfigure}{.5\textwidth}
			\includegraphics[width=.9\linewidth]{graph1.jpg}
			\caption{A subfigure}
			\label{fig:sub1}
		\end{subfigure}%
		\begin{subfigure}{.5\textwidth}
			\includegraphics[width=.9\textwidth]{graph1.jpg}
			\caption{A subfigure}
			\label{fig:sub2}
		\end{subfigure}
		\caption{A figure with two subfigures}
		\label{fig:test}
	\end{figure}
	\subsection{Figure}
	What a nice boat it is in Figure \ref{fig:boat}
	\begin{figure}[h]
		\centering
		\includegraphics[width=5cm]{boat.jpg}
		\caption{a picture of Boat }
		\label{fig:boat}
	\end{figure}
	\section{REFERENCE TEST}
	Referencing is one of the important parts of article writ-
	ing. Let's list some reference articles. This is\cite{Blum2005} a good
	conference paper. But, I like to read journal like\cite{Samarati:2001:PRI:627337.628183}. It is
	not a bad idea to read technical report like \cite{Li2011}.
	
	\section{Conclusion}
	\label{Conclusion}
	Again summarize you work to show that you have suc-
	cessfully achieved your objectives.
	
	\bibliographystyle{elsarticle-num}
	\bibliography{references}
	
\end{document}

