\documentclass[conference,one column]{IEEEtran}
\usepackage[utf8]{inputenc}
\usepackage[english]{babel}
\usepackage[T1]{fontenc}
\usepackage[utf8]{inputenc}
\usepackage{soul}
\usepackage{titlesec}
\usepackage{paralist}
\titleformat{\subsection}
  {\normalfont\fontsize{12}{17}\sffamily\bfseries\slshape}
  {\thesubsection}
  {1em}
  {}

\begin{document}
\title{Forensic Investigation Of Cloud Computing Environment}

\author{Dibbo Barua Chamak \\ Computer Science \& Engineering\\Roll:1603117\\RUET
\and
  Tabassum Rose \\ Computer Science \& Engineering\\Roll:1603118\\RUET
}

\maketitle
\begin{abstract}
One of the most important areas in the developing field of cloud computing is the way that investigators conduct researches in order to reveal the ways that a digital crime took place over the cloud.It is one of the most discussed information technologies today.So, there is an urgent need to extend the applications of investigation process in the cloud.In this paper, the related definition of cloud  computing along with computer forensics has been explained. furthermore, we compared traditional forensic investigations and forensics investigation in cloud and the concerns of users before using cloud computing have been judged.
\end{abstract}
\section*{Introduction}
The correct definition of cloud computing still has a cloudy appearance in I.T. world. But in a layman concern “ Cloud computing is the collection of some information, which resolves users query per its requirement ”.Perhaps the definition of cloud computing can be the process of accessing the data and related information as per end user requirement along with the services provided by the vendor chosen by end-user. The data related information could be anything like files or programs, etc.According to the definition by the National Institute of Standards and Technology (NIST), “ Cloud computing is a model which provides a convenient way of on-demand network access to a shared pool of configurable computing resources (e.g., networks, servers, storage, applications, and services), that can be rapidly provisioned and released with minimal management effort or service provider interaction” \cite{mell2009draft}. “ Cloud ” the term can be used as data hub where all the require data of user gets store and is delivered to the user whenever been asked by him. In today's era the computing world is heading towards the cloud computing technology and environment, with the matter of fact of its advancement.The advancement driven up by cloud computing environment is first:
 \begin{itemize}
 
  \item  99.99\% availability stream\cite{reilly2011cloud}.
  \item   Strong network infrastructure
   \item  Platform independency to run applications.
   \item   Powerful connectivity if internet
 \item  Gigantic data hub Cloud computing can even be defined as the further more extension of cluster computing that is cloud computing is inheritance of cluster computing along with it services. 
 
\end{itemize}
 Cloud computing has emerged as a popular and inexpensive computing paradigm in recent years. In the last 5 years alone, we have seen an explosion of applications of cloud computing technology, for both enterprises and individuals seeking additional computing power and more storage at a low cost. Small and medium scale industries find cloud computing highly cost effective as it replaces the need for costly physical and administrative infrastructure,and offers the flexible pay-as-you-go structure for payment.Khajeh-Hosseini et al. found that an organization could save 37\% cost if they would migrate their IT infrastructure from an outsourced data centre to the Amazon's Cloud\cite{khajeh2010cloud} .\\
 Now we discuss about the classification of cloud computing .Based on a cloud location, we can classify cloud as:
\begin{itemize}
\item Public,
\item Private,
\item Hybrid,
\item Community Cloud.
\end{itemize}
Based on a service that the cloud is offering, we classify as:
\begin{itemize}
\item IaaS (Infrastructure-as-a-Service),
\item PaaS(Platform-as-a-Service),
\item SaaS(Software-as-a-Service),
\item or, Storage, Database, Information, Process, Application, Integration, Security, Management, Testing-as-a-service.
\begin{itemize}
\bibliography{ref}
\bibliographystyle{ieeetr}
\end{document}
